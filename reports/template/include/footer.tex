
%----------------------------------------------------------------------------------------
%	DISCLAIMER
%----------------------------------------------------------------------------------------

\newpage

\thispagestyle{nofoot}

~\vfill




\textcolor{avansred}{\textbf{\fontsize{13pt}{19.5pt}\selectfont Organisatie Avans Hogeschool}}

Avans Hogeschool is in 2004 ontstaan door een fusie tussen Hogeschool Brabant en Hogeschool Brabant en 's-Hertogenbosch. Deze kennisinstituten werkten toen al nauw samen onder \'{e}\'{e}n Collega van Bestuur.

\

\textcolor{avansred}{\textbf{\fontsize{13pt}{19.5pt}\selectfont Academies}}

De opleidingen zijn verdeeld over 21 academies. Inclusief de Juridische Hogeschool Avans-Fontys. Wij hebben academies in de interessegebieden Economie en Bedrijf, Techniek, Gedrag en Maatschappij, Gezondheid, Exact en Informatica, Kunst en Cultuur, Recht en Bestuur, Onderwijs en Opvoeding, Aarde en milieu en Taal en Communicatie.

\

\textcolor{avansred}{\textbf{\fontsize{13pt}{19.5pt}\selectfont Onderzoek}}

Avans heeft 6 expertisecentra en ruim 25 lectoraten voor onderzoek. Groepen onderzoekers die praktijkgericht onderzoek doen.

\

\textcolor{avansred}{\textbf{\fontsize{13pt}{19.5pt}\selectfont Disclaimer}}

Aan dit rapport, waaronder ook de eventuele bijlagen worden bedoeld, kunnen geen rechten worden ontleend. Het is niet toegestaan om dit repport, geheel of gedeeltelijk, zonder toestemming te gebruiken of te verspreiden. Avans Hogeschool sluit elke aansprakelijkheid uit wanneer informatie in dit rapport niet correct, onvolledig of niet tijdig overkomt.

\

\

